\section{Accomplishments}
TODO: describe what was accomplished in the project\\
- gained knowledge for the devs\\
- list reusable sytem parts\\
- shortly describe functionality of the application\\
\section{What to do differently} % (fold)
\label{sec:Whattododifferently}
\subsubsection{Milestones} % (fold)
\label{ssub:Milestones}

% subsubsection Milestones (end)
\subsubsection{Unit tests} % (fold)
\label{ssub:Unittests}
Early in the development process we decided to not do unit tests.
At the time there were several reasons for that.
As the routing engine and the calendar integration were the first
things to be developed, they were crucial in the decision process.
Unit testing on the Android platform seemed rather complicated,
especially for code using the content provider backend. In a discussion
with our advisor, we decided that it would be too time consuming,
considering how much time already had passed for specification and
platform choosing. Therefore we decided to concentrate the time left
on actual development.

However, we had several situation in the development process, where
unit testing would have been useful. Many problems only became
apparent when certain elements were integrated together. Unit tests
could have revealed these much earlier. Especially as we developed
on our own much more often than anticipated. This accounted for more
problems whilst integrating, which could also have been minimized by
unit testing.

% subsubsection Unit tests (end)

\subsubsection{Bug tracker} % (fold)
\label{ssub:Bugtracker}
Although a bug tracker is provided by github, we didn't use it
until the very end of the project. In retrospect it would have been
better to utilize the bug tracker right from the start. This would
have forced us to look deeper into bugs and better document them, if
similar bugs appeared later.

Especially in combination with unit tests the bug tracker could have
improved the development process significantly. A good approach, for
example, would have been, to create a unit test for every occurring bug.
This means that there is an immediate way to test if the bug was
fixed and that it would not be introduced again later.

% subsubsection Bug tracker (end)
% section What to do differently (end)
\section{Future work}
TODO: describe possible followup projects:\\
- use system parts to build other mobile routing related application\\
- refine this application, make it market-ready\\
- improve system parts (especially routing / routing-performance)\\
