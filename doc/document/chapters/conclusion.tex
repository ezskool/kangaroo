\section{Accomplishments}
\label{sec:accomplishments}
\subsubsection{Application for optimized dayplans} % (fold)
\label{ssub:Application for optimized dayplans}
Several goals were accomplished while working on this teamproject. First of all
an application was created which supports a users day planning with the
following implemented functionalities:
\begin{itemize}
  \item Generate a dayplan, which is heavily optimized on completing as many
    tasks as possible
  \item Simple, yet powerful task management system
  \item Enter location data for any event in the calendar
  \item Full integration with Android calendar infrastructure
  \item Routing between events supporting a variety of time and space
    constraints
  \item Continous checking of the dayplan's consistency and compliance
  \item Notification of the user for every event and deadline according to the
    dayplan
\end{itemize}


\subsubsection{Easily reusable Open Source packages} % (fold)
\label{ssub:Easily reusable Open Source packages}
Additionally to the functionality, many parts of the application's structure can
be easily reused in other projects.

\paragraph{Calendar Library}
The calendar library provides an easy interface to query calendar data from the
Android platform. The queried data is then easily accessible via Java objects
and can be used in any Java application for Android. It can even be adapted to
other platforms with very small amount of adaptions to the library.

\paragraph{Routing Engine}
The routing engine underlying the kangaroo application is a fully self-contained
java package. It can be easily reused for routing purposes in any Java
application which needs to access Open Streetmap data.

\paragraph{Task Manager}
The task management functionality of kangaroo is also strongly encapsulated and
can therefore be reused easily in any Java project. The storage mechanism of
converting tasks into events and storing them at a specific date can also be
adapted easily to an SQLite database or any other preferred storage solution

\paragraph{Dayplan}
The Dayplan class as the centre of the whole application is also only loosely
coupled with the rest of the application. This means as it is also a pure Java
package, it can be reused easily and any other routing engine, calendar library,
task manager and user interface can be simply connected to the dayplan. Thus
providing the kangaroo functionality in a whole different environment and not
only on the android platform.

\subsubsection{Personal Experience} % (fold)
\label{ssub:Personal Experience}
Last but not least, a whole lot of knowledge was gained during the development
of kangaroo. At the beginning none of the developers was familiar with
developing applications for the Android platform. During the last months however
a huge knowledge was accumulated, including structure, optimization and
architecture of simple up to advanced applications for one of the big rising
mobile platforms.

% subsubsection Personal Experience (end)

\section{What to do differently} % (fold)
\label{sec:Whattododifferently}
\subsubsection{Milestones} % (fold)
\label{ssub:Milestones}
An important project management method we only utilized scarcely is setting
milestones. We defined some milestones for the most important parts of the
projects. However these were too few. The more precise definition of milestones
would have allowed us to wage the remaining and passed time in the project in a
more efficient manner and to assign tasks better. Therefore a better process of
creating and assigning milestones is one of the important things to do
differently next time.

% subsubsection Milestones (end)
\subsubsection{Unit tests} % (fold)
\label{ssub:Unittests}
Early in the development process we decided to not do unit tests.
At the time there were several reasons for that.
As the routing engine and the calendar integration were the first
things to be developed, they were crucial in the decision process.
Unit testing on the Android platform seemed rather complicated,
especially for code using the content provider backend. In a discussion
with our advisor, we decided that it would be too time consuming,
considering how much time already had passed for specification and
platform choosing. Therefore we decided to concentrate the time left
on actual development.

However, we had several situation in the development process, where
unit testing would have been useful. Many problems only became
apparent when certain elements were integrated together. Unit tests
could have revealed these much earlier. Especially as we developed
on our own much more often than anticipated. This accounted for more
problems whilst integrating, which could also have been minimized by
unit testing.

% subsubsection Unit tests (end)

\subsubsection{Bug tracker} % (fold)
\label{ssub:Bugtracker}
Although a bug tracker is provided by github, we didn't use it
until the very end of the project. In retrospect it would have been
better to utilize the bug tracker right from the start. This would
have forced us to look deeper into bugs and better document them, if
similar bugs appeared later.

Especially in combination with unit tests the bug tracker could have
improved the development process significantly. A good approach, for
example, would have been, to create a unit test for every occurring bug.
This means that there is an immediate way to test if the bug was
fixed and that it would not be introduced again later.

% subsubsection Bug tracker (end)
% section What to do differently (end)
\section{Future work}
As future projects, one can imagine many fields, not only directly concerning
the kangaroo application. Of course, some minor adjustments have to be done to
kangaroo, if it should be ready to be deployed via the Android market. The UI
could be polished a bit to match the design of modern mobile applications.
Additionally some system parts can be improved, mainly in the area of routing
performance, to guarantee a smother and faster experience for the end user.

Nevertheless one of the big strengths of the kangaroo project are definitely the
amount of easily, reusable Java packages. As the whole code is open sourced, all
parts can be used to build other routing related applications. But also
integrate with calendar data, build an improved task management application or
port the Dayplan optimizing functionality to another routing engine like Google
Maps for example.
