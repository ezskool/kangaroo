All data integration on the Android plattform is done via
so called ``content providers''. These providers are the
only way to transfer data between applications, since there
is no shared space which all applications could use.

However the content provider for the calendar data is
working, but not specified in the official developers
documentation. That means that the API could change at
any time, which has to be considered at the application
design.

In order to not be too dependent on the changing API, a
wrapper layer was built around it. Now there are only four
methods which have to be adapted if the API changes.
\begin{itemize}
  \item queryEvents
  \item insertEventToBackend
  \item updateEventInBackend
  \item deleteEventFromBackend
\end{itemize}

The ``queryEvents'' method can be given a SQL statement as a selection
statement and a String array. Values in the array subsequently replace all
the ``?'' in the statement. The method then returns an ArrayList of
CalendarEvent objects, which suited the selection.

The ``insertEventToBackend'' method is used to insert new events via the
content provider. It takes a CalendarEvent object, builds an Android
ContentValues object from it and inserts this into the backend.

The ``updateEventInBackend'' method does practically the same, but updates
the events in the backend, instead.

At last the ``deleteEventFromBackend'' method removes the passed CalendarEvent
completely from the backend.

These four methods, described above, are the sole fundament of the calendar
wrapper library. As they mostly consist of less than ten lines of code,
the adaption to a new API can be done extremely easily.

