One essential feature of our application is its capability to not only consider calendar items for itself but to put them in their geographical context and to account for local traffic facilities. This implies the need for a set of data representing these facilites and an engine to operate on the data.\newline

From a very abstract and general perspective, this set of data can be divided into two categories, namely a dynamic and a static one. As static, in these terms, one should label all facilities which do not depend on any schedule and may in principle by used at any time (for example streets and highways). Dynamic facilities are the ones, that have a more or less fixed schedule constraining their usage (for example tramways and public bus service).\newline

In this project we will only consider static facilities, since also accounting for dynamic ones requires routing algorithms that go far beyond standard algorithms resulting in a more complex application structure and a much more complex routing engine. Besides this aspect, it is a big issue to gather scheduling data about dynamic traffic facilities as there would be a number of carriers with different interfaces to include. However, the user might notably benefit from the abolishment of this restriction. It might be part of future work to include this feature.\newline
 
As potential sources for static traffic facility data, one finds basically two options

\begin{itemize}
 
	\item \textbf{Google Maps}
	
		Google Maps is an online navigation and routing service. Its map material has high quality but in some areas lacks actuality. Its main drawback may anyway be found in the absence of a free simple-to-use offline interface. 
	
	\item \textbf{Openstreetmap}
	
		Openstreetmap is a free and open database for a world wide map. Everyone is free to add, change and remove map items. Consequently its map material varies over a hugh scale in quality, density and actuality. It can be used online and offline by downloading an Openstreetmap XML file of a rectangular map area. This file contains every map item of this area.
  
\end{itemize} 

Besides the two options given above there are of course lots of other providers, but either not free of charge or not very popular. Consequently, Openstreetmap seems to be the one to choose for our project. The following will briefly outline its data scheme.\newline

\subsubsection{Data scheme of Openstreetmap}

Openstreetmap uses three main entities to map traffic facilities and geographical characteristics:

\begin{itemize}

	\item \textbf{Nodes}
	
		Nodes are the fundamental elements setting up model points for direct use or as a basis for superior map elements. A node imperatively has a unique ID and values for latitude and longitude, but can have optional parameters provided as tags with key-value-pairs. Adding tags to a node is the way to define, for example, Points Of Interest (POI).
		
	\item \textbf{Ways}
	
		Ways are made up of an ordered list of nodes which span the way in the given order. Just like a node, a way must have a unique\footnote{unique with respect to other ways; there still might be a node with the same ID} ID and may have optional tags.\newline
		As a way can either be closed\footnote{first and last nodes are the same} or open and can have tags to specifiy parameters, it can be used to map streets, highways, roundabouts, buildings or even administrative areas.	
		
	\item \textbf{Relations}
	
		Realtions provide a simple way to group other map elements. This feature will not be used.
	
\end{itemize}
