The following chapter will give a more detailed introduction in the structure of Traveling Salesman.\newline

As already mentioned, Traveling Salesman is designed following a modular architecture. Every essential or optional type of module is defined by a java interface with basic methods to provide its desired functionality. The most important ones in our project are

\begin{itemize}

	\item Interface \texttt{IDataSet}
	
		\texttt{IDataSet} provides methods to store and receive Openstreetmap map elements. Identification of elements to receive can either be done by unique Openstreetmap ID or by parameters to search for. This, for example, may be the name of the element or simply its distance to a specific point, that has to fall below a cetain value.\newline
		 In dependence of its implementation, data storage is done in memory (\texttt{MemoryDataSet}), on disc or in a database.
	
	\item Interface \texttt{IRouter}

		\texttt{IRouter} is the interface which has to be implemented by any routing class. Its primary task is triggered by calling the \texttt{route(...)} method, which performs routing between a point to start from an one or more destination points. One has to pass an \texttt{IDataSet} object specifying routing resources, when calling \texttt{route(...)}. The router is to return a \texttt{Route} object containing detailed information about the path.
	
	\item Interface \texttt{IRoutingMetric}
	
		As to give the router a  measure of routing costs, one has to define a class implementing the \texttt{IRoutingMetric} interface, which allows the router to query routing costs of particular routing steps.\newline
		The default implementation (\texttt{ShortestRouteMetric}) simply measures distance between start and end of a routing step. One might feel uncomfortable with this, because it does not privilege any type of street and hence the shortest route is probably not the fastest one. However, this can be overcome with a re-implementation of a routing metric.
	
\end{itemize}

To represent Openstreetmap map elements (nodes, ways and relations), Traveling Salesman uses classes provided by Osmosis, which can be considered as a Java based framework for processing of Openstreetmap data. Osmosis employs classes named \texttt{Node}, \texttt{Way} and \texttt{Relation} to represent Openstreetmap map elements accordingly. 